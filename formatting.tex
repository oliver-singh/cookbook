%------------------------------------------------
%   Mathrule
%------------------------------------------------

\makeatletter

\newcommand\mathrule[3][0pt]{%
  \ifdim#2>#3\math@hrule[#1]{#2}{#3}\else\math@vrule[#1]{#2}{#3}\fi}

\newcommand\math@hrule[3][0pt]{%
  \gdef\mystery@factor{0.07}%
 \@tempdima=#3%
  \rule[#1]{0pt}{#3}% Needed to account for .5\@tempdima vertical offset of rounded rule
  \raisebox{.5\@tempdima+#1}{%
    \makebox[#2][l]{\kern-.5\@tempdima\@@mathrule{#2}{#3}}}%
}

\newcommand\math@vrule[3][0pt]{%
  \gdef\mystery@factor{0.0}%
 \@tempdima=#2%
  \rule[#1]{0pt}{#3}% Needed to account for .5\@tempdima vertical offset of rounded rule
  \raisebox{-.0\@tempdima+#1}{%
    \kern0.5\@tempdima%
    \rotatebox{90}{\kern-0.5\@tempdima\makebox[#3][l]{\@@mathrule{#3}{#2}}}%
    \kern0.5\@tempdima}%
}

\def\@@mathrule#1#2{%
  \@tempdimb=#2%
  \@tempdima=\dimexpr#1-\mystery@factor\@tempdimb%Why 0.07 for \math@hrule?
  \pdfliteral{%
    q []0 d %
    1 J %  set line cap to rounded ends
    \strip@pt\@tempdimb\space w \strip@pt\@tempdimb\space 0 m %
    \strip@pt\@tempdima\space 0 l S Q }}
\makeatother

%------------------------------------------------
%   Introduction
%------------------------------------------------
\newcommand{\introduction}[1]
{% Introduction before the preparation and ingredients
    \def\cookintroduction
    {
	\vspace{-37pt}
        \begin{center}
            #1
        \end{center}
    }
}

%------------------------------------------------
%   Invisible subsections
%------------------------------------------------
\newcommand{\invisiblesubsection}[1]{%
  \refstepcounter{subsection}%
  \addcontentsline{toc}{subsection}{#1}}


%------------------------------------------------
%   Ingredients
%------------------------------------------------
\newcommand{\ingredients}[1]
{% Introduction before the preparation and ingredients
    \def\cookingredients
    {
	\begin{tabular}{@{}r l@{}}
	\multicolumn{2}{@{}l@{}}{\large\bf Ingredients}\\[1ex]
	#1
	\end{tabular}
		
    }
}

\newcommand{\ingredient}[2]
{
#1  & #2 \\
}

\newcommand{\ingredientsubtitle}[1]
{
& \\[\dimexpr-\normalbaselineskip+0.5ex]
\multicolumn{2}{@{}l@{}}{\bf #1} \\
}

\newcommand{\optional}
{
(\textit{optional})
}

%------------------------------------------------
%   Preperation
%------------------------------------------------
\newcounter{step}
\newcommand{\preperation}[1]
{% Introduction before the preparation and ingredients
    \def\cookpreperation
    {
	#1
    }
	\setcounter{step}{0}
}
\newcommand{\step}{%
    \par
    \stepcounter{step}  % <-- moved here, since shouldn't be in the argument of lettrine
    \lettrine[%
        lines=2,
        lhang=0,          % space into margin, value between 0 and 1
        %loversize=0.15,   % enlarges the height of the capital
        slope=0em,
        findent=0.5em,      % gap between capital and intended text
        nindent=0em       % shifts all intended lines, beginning with the second line
    ]{\thestep}{}%
}


%------------------------------------------------
%   Recipes
%------------------------------------------------

\newenvironment{recipe}[1]{
\def\cookintroduction{}
\def\cookingredients{}
\def\cookpreperation{}

\def\cookrecipetitle{#1}
}
{
\invisiblesubsection{\cookrecipetitle}
\begin{center}
\Large\bf\MakeUppercase{\cookrecipetitle}
\mathrule[14pt]{\textwidth}{1.5pt}
\end{center}
\cookintroduction
%\begin{tabular}{@{}l l@{}}
\begin{multicols}{3}
	
\cookingredients{}
\columnbreak

\cookpreperation{}
\end{multicols}

}

\widowpenalties 1 10000
\raggedbottom