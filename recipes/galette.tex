\begin{recipe}{Peach Gillette}
\introduction{You can swap the peaches for plums, pears, or apples.}
\serves{6}
\oventemp[180]{200}
\ingredients{
\ingredientsubtitle{Pastry}
\ingredient{\unit[175]{g}}{flour}
\ingredient{\unit[120]{g}}{unsalted butter}
\ingredient{\unit[1/2]{tsp}}{salt}
\ingredient{\unit[1/2]{tbsp}}{sugar}
\ingredient{\unit[2]{tbsp}}{ground almonds}
\ingredient{up to \unit[50]{ml}}{ ice-cold water}
\ingredient{}{egg white}

\ingredientsubtitle{Filling}
\ingredient{4}{peaches}
\ingredient{\unit[50]{g}}{brown sugar}
\ingredient{\unit[3]{tbsp}}{corn flour}
}
\preperation{
\step Cut the butter into small cubes, mix with the flour, almonds, salt and sugar in a large bowl. Rub butter in with fingertips (or just put it all in a food processor). Add water a spoon at a time until it just clumps together (you may not need any).

\step Form the pastry into a ball, wrap in cling film, and put it into the fridge. It is best to leave the pastry in the fridge for a couple of hours if possible.

\step Cut the peaches into \unit[1/2]{cm} slices, and combine with sugar and corn flour. If some liquid accumulates at the bottom then pour it off. Add sugar to taste, and add corn flour if the mixture looks too wet.

\step Heat the oven to \unit[200]{\textcelcius}. On a floured surface roll the pastry out into a large circle around a pound coin thick. This should be around \unit[30]{cm} in diamater.

\step Transfer the pastry onto a lined baking tray. Place the peaches on top leaving a border around the edges. Fold over the overhanging edges, brush with egg and sprinkle with sugar.

\step Bake for 30 minutes, or until the edges are brown and the peaches are cooked.

\step To serve, sprinkle with icing sugar, or glaze. Serve with vanilla ice cream.
}
\end{recipe}